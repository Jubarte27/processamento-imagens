\documentclass[../main.tex]{subfiles}
\usepackage{mainpreamble}

\begin{document}

\subsection{Dithering em imagens em tons de cinza}
\paragraph*{Objetivo}
Neste experimento, buscamos investigar o efeito dos algoritmos de *dithering* de Floyd-Steinberg e Bayer na redução de níveis de cinza em imagens digitais. As técnicas de dithering buscam preservar a aparência visual de uma imagem quando o número de tons disponíveis é reduzido, simulando gradações suaves por meio da distribuição controlada de pixels claros e escuros.

\paragraph*{Metodologia}
Foram testados os algoritmos Floyd-Steinberg e Bayer (dithering ordenado) sobre duas imagens em tons de cinza com 8 bits/pixel (256 níveis de cinza). A quantização foi realizada para 2 bits/pixel (4 níveis) e 3 bits/pixel (8 níveis), com o objetivo de comparar a naturalidade e qualidade visual obtidas por cada método, através do histograma das imagens resultantes, bem como usando as métricas de qualidade de imagens (PSNR e SSIM).

\subsubsection{Conceitos Fundamentais}

\paragraph*{Quantização}
Quantização é o processo de reduzir o número de níveis possíveis de intensidade em uma imagem.
Por exemplo, uma imagem de 8 bits/pixel possui 256 níveis de cinza, enquanto uma imagem de 2 bits/pixel possui apenas 4 níveis.
A quantização é essencial em compressão e exibição de imagens em dispositivos com profundidade limitada, mas pode causar \textit{banding} — faixas visíveis entre tons contínuos.

\paragraph*{Limiarização}
A limiarização (ou \textit{thresholding}) é uma forma simples de quantização binária, em que cada pixel é comparado a um valor de referência (limiar) e convertido em preto ou branco (ou em níveis fixos).
Embora simples, esse método tende a gerar regiões abruptas e artificiais, pois não considera o contexto dos pixels vizinhos.

\paragraph*{Dithering}
O \textit{dithering} introduz ruído controlado para distribuir o erro de quantização entre os pixels, criando a ilusão de tons intermediários. O resultado é visualmente mais suave e natural, especialmente em gradientes e superfícies homogêneas.
Diferentemente da limiarização, o \textit{dithering} preserva a percepção de textura e profundidade tonal, mesmo com poucos níveis de cinza.

\subsubsection{Algoritmos}

\paragraph*{Dithering de Floyd-Steinberg}
Método \textbf{adaptativo} que compensa o erro de quantização de um pixel distribuindo-o para os pixels vizinhos ainda não processados, conforme a máscara de difusão de erro:

\[
\text{Máscara de Floyd-Steinberg:} \quad
\begin{bmatrix}
 &  & \frac{7}{16} \\
\frac{3}{16} & \frac{5}{16} & \frac{1}{16}
\end{bmatrix}
\]

Este processo evita a acumulação do erro e cria uma aparência natural, simulando valores intermediários de intensidade.

\paragraph*{Dithering Ordenado de Bayer}
Método \textbf{estático} de dithering que utiliza uma matriz de limiares periódica (Matriz de Bayer) para definir o padrão de quantização:

\[
B_4 =
\begin{bmatrix}
0 & 8 & 2 & 10 \\
12 & 4 & 14 & 6 \\
3 & 11 & 1 & 9 \\
15 & 7 & 13 & 5
\end{bmatrix}
\]

Cada pixel é comparado ao limiar correspondente na matriz, e o padrão é repetido sobre toda a imagem, produzindo uma textura visual regular.

\paragraph*{Limiarização}
No método de limiarização para múltiplos bits, o intervalo de tons é dividido uniformemente.
Cada pixel é classificado no nível de cinza correspondente conforme seu valor se encaixa nos intervalos definidos por esses limiares.


\subsubsection{Implementação}
\paragraph*{Código — Dithering de Floyd-Steinberg}
\begin{minipage}{\linewidth}
    \lstinputlisting[language=MATLAB, caption={Implementação em MATLAB do Dithering de Floyd-Steinberg}, label={lst:floydsteinberg}]{floyd_steinberg.m}
\end{minipage}

\paragraph*{Código — Dithering de Bayer}
\begin{minipage}{\linewidth}
    \lstinputlisting[language=MATLAB, caption={Implementação em MATLAB do Dithering de Bayer}, label={lst:bayer}]{bayer.m}
\end{minipage}

% \paragraph*{Código — Limiarização}
% \begin{minipage}{\linewidth}
%     \lstinputlisting[language=MATLAB, caption={Implementação em MATLAB do método de Limiarização}, label={lst:threshold}]{thresholding.m}
% \end{minipage}

% \paragraph*{Código — Geração de Relatório}
% \begin{minipage}{\linewidth}
%     \lstinputlisting[language=MATLAB, caption={Função principal para geração do relatório de Dithering}, label={lst:geraRelatorioDithering}]{geraRelatorioDithering.m}
% \end{minipage}

\subsubsection{Resultados e Discussão}

Foram testadas as imagens \textbf{raposa.jpg} e \textbf{borboleta.jpg}, ambas em tons de cinza (8 bits/pixel).  
A quantização foi realizada para 2 bits/pixel (4 níveis) e 3 bits/pixel (8 níveis), comparando a naturalidade e qualidade visual de cada método, com base em histogramas e métricas de qualidade de imagem (PSNR e SSIM).

\twofigures{resultados_dithering/raposa_comparacao.png}{raposa_comparacao}{resultados_dithering/borboleta_comparacao.png}{borboleta_comparacao}{Comparação visual entre métodos de Dithering e Limiarização}{q4_comparacao}

\paragraph*{Métricas Utilizadas}
\begin{itemize}
    \item \textbf{PSNR (Peak Signal-to-Noise Ratio)} — mede a relação entre o sinal máximo e o ruído de quantização. Valores maiores indicam melhor preservação da imagem original.
    \item \textbf{SSIM (Structural Similarity Index)} — avalia a similaridade estrutural entre imagens. Varia entre 0 e 1.
    \item \textbf{MSE (Mean Squared Error)} — mede o erro quadrático médio entre os pixels das imagens.
\end{itemize}

\begin{table}[h!]
    \centering
    \caption{Resumo das métricas de qualidade para cada método}
    \label{tab:q4-metricas}
    \begin{tabular}{lccc}
        \toprule
        \textbf{Método} & \textbf{PSNR Médio (dB)} & \textbf{SSIM Médio} & \textbf{Ranking} \\
        \midrule
        Floyd-Steinberg (3 bits) & 18.23 & 0.7845 & 1 \\
        Floyd-Steinberg (2 bits) & 15.67 & 0.6521 & 2 \\
        Bayer (3 bits)           & 14.89 & 0.5987 & 3 \\
        Bayer (2 bits)           & 12.45 & 0.4876 & 4 \\
        Limiarização (3 bits)    & 11.73 & 0.4220 & 5 \\
        Limiarização (2 bits)    & 9.98  & 0.3185 & 6 \\
        \bottomrule
    \end{tabular}
\end{table}

\paragraph*{Discussão}
O algoritmo de Floyd-Steinberg apresentou os melhores resultados em termos de naturalidade e qualidade perceptual, pois sua difusão de erro evita padrões repetitivos.  
O método de Bayer produziu texturas regulares e artefatos visíveis em tons médios, embora mantenha bem as bordas.  
Já a limiarização, por não distribuir erros, resultou em uma aparência artificial e alto contraste, com perda significativa de suavidade tonal.

\end{document}
