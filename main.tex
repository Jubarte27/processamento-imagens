\documentclass{article}
\usepackage[utf8]{inputenc}
\usepackage{graphicx} % Required for inserting images
\usepackage{float}  % Allows use of [H] float placement
\usepackage[brazilian]{babel}   %   pt-br
\usepackage{bookmark}
\usepackage{booktabs}   % For \toprule, \midrule, \bottomrule, \addlinespace
\usepackage[left=3cm,right=3cm]{geometry} % Increase horizontal margins
\usepackage{tabularx}   % For automatic column wrapping to text width
\usepackage{hyperref}   % For \url and hyperlinks
\usepackage[nottoc,numbib]{tocbibind} % For making the bibliography appear in the table of contents
\usepackage{titlesec} % For paragraph to behave like sub-sub-sub-section

\usepackage{float}
\usepackage{caption}
\usepackage{subcaption}
\usepackage{xparse}
\usepackage{calc}

%% Pretty code
\usepackage{listings}   % For adding text files
\usepackage{matlab-prettifier}

% Define custom colors
\definecolor{codegreen}{rgb}{0,0.6,0}
\definecolor{codegray}{rgb}{0.5,0.5,0.5}
\definecolor{codepurple}{rgb}{0.58,0,0.82}
\definecolor{backcolor}{rgb}{0.95,0.95,0.92}


\newcommand{\twofigures}[6]{
\begin{figure}[H]
\centering
\begin{subfigure}{.5\textwidth}
  \centering
  \includegraphics[width=.9\linewidth]{#1}
  \label{fig:#2}
\end{subfigure}%
\begin{subfigure}{.5\textwidth}
  \centering
  \includegraphics[width=.9\linewidth]{#3}
  \label{fig:#4}
\end{subfigure}
\caption{#5}
\label{fig:#6}
\end{figure}
}


\ExplSyntaxOn
\clist_new:N \l_figures_names
\newcommand{\figurescompare}[8]{
    \clist_set:Nn \l_figures_names {#1}
    \begin{figure}[H]
        \centering
        \begin{subcaptionblock}{.5\textwidth-.5em}
            \centering
            \clist_map_inline:Nn \l_figures_names {
                \includegraphics[width=1\linewidth]{#2##1.jpg}
                \vskip .5em
            }
            \caption{#5}
        \end{subcaptionblock}%
        \hskip .5em
        \begin{subcaptionblock}{.5\textwidth-.5em}
            \centering
            \clist_map_inline:Nn \l_figures_names {
                \includegraphics[width=1\linewidth]{#3##1#4.png}
                \vskip .5em
            }
            \caption{#6}
        \end{subcaptionblock}%
        \caption{#7}
        \label{fig:#8}
    \end{figure}
}

\ExplSyntaxOff

% \lstset{style=Matlab-bw} %% Melhor para imprimir
\lstset{style=Matlab-editor} %% Melhor para ver no pdf

\hypersetup{hidelinks}  % Remove colored boxes around links

% For paragraph to behave like sub-sub-sub-section
\setcounter{secnumdepth}{4}

\titleformat{\paragraph}
{\normalfont\normalsize\bfseries}{\theparagraph}{1em}{}
\titlespacing*{\paragraph}
{0pt}{3.25ex plus 1ex minus .2ex}{1.5ex plus .2ex}

\title{\textbf{Processamento de imagens}}
\author{
    Enzo Borges Segala \and
    Jonathan Gabriel Nunes Mendes \and
    Matheus Machado Cezar \and
    Marcos Henrique Volpato de Moraes \and
    Miguel Predebon Abichequer \and
    Nicolas Rosenthal Dal Corso \and
    Rafael Silveira Bandeira
}
\date{11/11/2025}

\begin{document}

\maketitle

\newpage
\tableofcontents
\newpage

\section{Introdução}
...

\section{Tarefas}

\subsection{Interpolação em imagens coloridas}
\subsubsection{Arquivo principal}
\begin{minipage}{\linewidth}
    \lstinputlisting[language=MATLAB]{q1/q1.m}
\end{minipage}

\subsubsection{Funções auxiliares}
\begin{minipage}{\linewidth}
    \lstinputlisting[language=MATLAB]{q1/bicubic_channel.m}
\end{minipage}
\begin{minipage}{\linewidth}
    \lstinputlisting[language=MATLAB]{q1/bicubic_interpolate.m}
\end{minipage}
\begin{minipage}{\linewidth}
    \lstinputlisting[language=MATLAB]{q1/bicubic_resize.m}
\end{minipage}
\begin{minipage}{\linewidth}
    \lstinputlisting[language=MATLAB]{q1/bilinear_resize.m}
\end{minipage}
\begin{minipage}{\linewidth}
    \lstinputlisting[language=MATLAB]{q1/nearest_neighbour_resize.m}
\end{minipage}

\subsection{Realce de imagens no domínio espacial (da imagem)}
\subsubsection{Arquivo principal}
\begin{minipage}{\linewidth}
    \lstinputlisting[language=MATLAB]{q2/q2.m}
\end{minipage}


\subsection{Filtragem espacial}
\subsubsection{Arquivos principais}
\paragraph{Ruídos Salt \& Pepper e Gaussian}

% Imagens Originais
\twofigures{q3/raposa.jpg}{raposa}{q3/borboleta.jpg}{borboleta}{Imagens originais}{q3_original}

\paragraph*{Adição de Ruídos}
Para iniciar esta etapa, deve-se inserir diferentes níveis de ruído sal e pimenta (salt and pepper) e ruído Gaussiano simultaneamente nas imagens selecionadas: “raposa.jpg” e “borboleta.jpg”. Para isso, utiliza-se a função imnoise, responsável por contaminar a imagem com ambos os tipos de ruídos.

\begin{minipage}{\linewidth}
    \lstinputlisting[language=MATLAB]{q3/adicao_de_ruido.m}
\end{minipage}

Como resultado, temos as imagens originais contaminadas com ruído Gaussiano de variância 0.01 e com ruído salt \& pepper de densidade 0.05.

\twofigures{q3/.out/raposa_ruido.png}{raposa_ruido}{q3/.out/borboleta_ruido.png}{borboleta_ruido}{Imagens com ruído}{q3_ruido}

\paragraph*{Filtragem Alpha Trimmed Mean}
Em seguida, as imagens passam pelo filtro Alpha Trimmed Mean. Para isso, foi desenvolvida a função alpha\_trimmed\_mean\_filter 3x3, que aplica o filtro aos três canais de cor: vermelho, verde e azul, e, posteriormente, os resultados são combinados novamente para gerar a imagem final com os ruídos atenuados.

\begin{minipage}{\linewidth}
    \lstinputlisting[language=MATLAB]{q3/alpha_trimmed_mean_filter.m}
\end{minipage}

\twofigures{q3/.out/raposa_filtered.png}{raposa_filtered}{q3/.out/borboleta_filtered.png}{borboleta_filtered}{Imagens filtradas}{q3_filtered}

\paragraph*{Medições PSNR e SNR}
A avaliação da qualidade da filtragem é feita utilizando os parâmetros SNR (Signal-to-Noise Ratio) e PSNR (Peak Signal-to-Noise Ratio), por meio de suas respectivas funções. Como esses indicadores medem a relação entre o sinal original e o ruído, valores mais altos de SNR ou PSNR correspondem a menor presença de ruído, indicando melhor qualidade da imagem.

\begin{minipage}{\linewidth}
    \lstinputlisting[language=MATLAB]{q3/maina.m}
\end{minipage}

\begin{center}
\begin{tabular}{ c c c }
                  & Borboleta& Raposa  \\
    Filtrada PSNR & 26.16 dB & 24.87 dB \\
    Filtrada SNR  & 19.84 dB & 19.14 dB \\
    Ruidosa PSNR  & 16.26 dB & 16.58 dB \\
    Ruidosa SNR   & 10.25 dB & 11.14 dB  \\
\end{tabular}
\end{center}


\paragraph{Unsharp Masking}

Ao aplicar este filtro, os detalhes da imagem original são realçados sem a introdução de ruído. Inicialmente, a imagem é suavizada por meio de convoluções Gaussianas, que funcionam como filtros passa-baixa, removendo componentes de alta frequência, como texturas finas e bordas. Em seguida, a imagem suavizada é subtraída da original, e os detalhes resultantes são ampliados através da multiplicação por um fator de ganho “amount”. Por fim, esse resultado é somado à imagem original, proporcionando o realce final de detalhes e contornos.

OBS.:  2 parâmetros no código:
Sigma: Define o grau de suavização na máscara Gaussiana;
Fator de ganho(amount): Define quanto os detalhes serão amplificados;

\begin{minipage}{\linewidth}
    \paragraph*{Código}
    \lstinputlisting[language=MATLAB]{q3/mainb.m}
\end{minipage}

\paragraph*{Comparações}
Processando as imagens com sigma = 2.5 e amount = 0.5. Apesar do valor relativamente alto de sigma, as alterações não são muito evidentes devido ao baixo valor de amount. Ainda assim, é possível notar que os detalhes maiores, como os detalhes das asas e o pelo, apresentam maior definição e nitidez.

\figurescompare{borboleta,raposa}{q3/}{q3/.out/}{_25_05_unsharp}{Original}{Com Unsharp Masking}{sigma=2.5 e amount=0.5}{q3_25_05_unsharp}

Processando as imagens com sigma = 2.5 e amount = 2.5. Nesse caso, as alterações tornam-se mais evidentes, com as asas mais nítidas e menos borradas. Na raposa, a pelagem também apresentou maior definição. 

\figurescompare{borboleta,raposa}{q3/}{q3/.out/}{_25_25_unsharp}{Original}{Com Unsharp Masking}{sigma=2.5 e amount=2.5}{q3_25_25_unsharp}

Para avaliar as alterações geradas por diferentes valores de sigma, mantivemos amount = 2.5 para intensificar a máscara. No primeiro experimento, com sigma = 0.5, observa-se praticamente nada de mudança, somente um realce de pequenos detalhes.

\figurescompare{borboleta,raposa}{q3/}{q3/.out/}{_05_25_unsharp}{Original}{Com Unsharp Masking}{sigma=0.5 e amount=2.5}{q3_05_25_unsharp}

\subsection{Dithering}
\subsubsection{Arquivo principal}
\begin{minipage}{\linewidth}
    \lstinputlisting[language=MATLAB]{q4/q4.m}
\end{minipage}


\subsection{Operacoes Geométricas}
\subsubsection{Arquivo principal}
\begin{minipage}{\linewidth}
    \lstinputlisting[language=MATLAB]{q5/q5.m}
\end{minipage}


\end{document}


