\documentclass[../main.tex]{subfiles}
\usepackage{mainpreamble}

\begin{document}

\subsection{Operacoes Geométricas}
Tanto para a escala quanto para a rotação de imagens, a interpolação do vizinho mais próximo produz resultados mais quadrados, mantendo bem a variação de cores da imagem original. Já a interpolação bilinear perde um pouco a definição das cores, mas produz resultados mais suaves, evitando as bordas serradas geradas pelo vizinho mais próximo.

Para a escala de imagens em mais de 3x, a interpolação bilinear é a que consegue o resultado mais próximo da original; contudo, a interpolação pelo vizinho mais próximo ainda tem seu uso em videogames que utilizam imagens propositalmente pixeladas, como no caso das pixel arts.

\begin{figure}[H]
    \centering
    \begin{subfigure}{.5\textwidth}
        \centering
        \scalegraphics{.out/lena.png}
    \end{subfigure}%
    \caption{Lena original}
    \label{fig:q5_lena_original}
\end{figure}

\newpage
\subsubsection{Resultados}
\twofiguresnoupscale{.out/lena_2X_nn_section.png}{Nearest Neighbour}{.out/lena_2X_bl_section.png}{Bilinear}{Zoom 2X}{q5_lena_2x}
\twofiguresnoupscale{.out/lena_3X_nn_section.png}{Nearest Neighbour}{.out/lena_3X_bl_section.png}{Bilinear}{Zoom 3X}{q5_lena_3x}

\twofiguresnoupscale{.out/lena_2X_nn_r5_section.png}{Nearest Neighbour}{.out/lena_2X_bl_r5_section.png}{Bilinear}{Zoom 2X - Rotacionado 5°}{q5_lena_2x_r5}
\twofiguresnoupscale{.out/lena_3X_nn_r5_section.png}{Nearest Neighbour}{.out/lena_3X_bl_r5_section.png}{Bilinear}{Zoom 3X - Rotacionado 5°}{q5_lena_3x_r5}

\twofiguresnoupscale{.out/lena_2X_nn_r45_section.png}{Nearest Neighbour}{.out/lena_2X_bl_r45_section.png}{Bilinear}{Zoom 2X - Rotacionado 45°}{q5_lena_2x_r45}
\twofiguresnoupscale{.out/lena_3X_nn_r45_section.png}{Nearest Neighbour}{.out/lena_3X_bl_r45_section.png}{Bilinear}{Zoom 3X - Rotacionado 45°}{q5_lena_3x_r45}

\newpage
\subsubsection{Código}
\lstinputlisting{escala.m}

\end{document}