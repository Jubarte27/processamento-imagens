\documentclass[../main.tex]{subfiles}
\usepackage{mainpreamble}

\begin{document}

\subsection{Interpolação em imagens coloridas}

\begin{minipage}{\linewidth}
    \lstinputlisting{q1.m}
\end{minipage}

\twofiguresnoupscale{.out/cat.png}{cat}{.out/hamster.png}{hamster}{Imagens originais}{q1_original}

\paragraph*{Algoritmos}
Foram escolhidos os algoritmos: Vizinho mais próximo (orderm 0), Bilinear (ordem 1) e Bicúbico (ordem 3)

\subsubsection{Interpolação por vizinho mais próximo}
Pode-se dizer que a principal vantagem deste algoritmo é a simplicidade da sua implementação e, por conta dela, chega ao resultado mais rápido. Contudo, seu resultado é uma imagem bastante pixelizada, o que torna inadequada a sua utilização em fotografias. Apesar disso, ainda é bastnte utilizada por artistas que trabalham com pixel art, já que neste estilo é desejado ver cada pixel.

\figurescomparenoupscale{cat,hamster}{_nearest_neighbour}{Original}{Interpolado}{Vizinho mais Próximo}{q1_nearest_neighbour}
\paragraph{Código}
\begin{minipage}{\linewidth}
    \lstinputlisting{nearest_neighbour_resize.m}
\end{minipage}

\subsubsection{Bilinear}
Este algoritmo suaviza as transições entre pixels, o que gera um resultado visual mais natural que o vizinho mais próximo. Contudo, pela simplicidade da interpolação, causa leve borramento e perda de nitidez.

\figurescomparenoupscale{cat,hamster}{_bilinear}{Original}{Interpolado}{Bilinear}{q1_bilinear}
\paragraph{Código}
\begin{minipage}{\linewidth}
    \lstinputlisting{bilinear_resize.m}
\end{minipage}
\begin{minipage}{\linewidth}
    \lstinputlisting{lerp.m}
\end{minipage}

\subsubsection{Bicúbico}
Dentre os 3 algoritmos estudados é o que produz melhor resultado, gerando imagens finais mais suaves e detalhadas. Ainda existe perda de nitidez, mas ela é mínima em comparação. É também um algoritmo mais complexo e mais lento que os demais, além de gerar pequenos artefatos em bordas com o contraste muito alto.

\figurescomparenoupscale{cat,hamster}{_bicubic}{Original}{Interpolado}{Bicúbico}{q1_bicubic}
\paragraph{Código}
\begin{minipage}{\linewidth}
    \lstinputlisting{bicubic_resize.m}
\end{minipage}

\subsubsection{Funções auxiliares}
\begin{minipage}{\linewidth}
    \lstinputlisting{at_dim.m}
\end{minipage}
\begin{minipage}{\linewidth}
    \lstinputlisting{get_at_dim.m}
\end{minipage}
\begin{minipage}{\linewidth}
    \lstinputlisting{set_at_dim.m}
\end{minipage}

\begin{minipage}{\linewidth}
    \lstinputlisting{docked_figure.m}
\end{minipage}
\begin{minipage}{\linewidth}
    \lstinputlisting{imgs_in_docked_figures.m}
\end{minipage}

\begin{minipage}{\linewidth}
    \lstinputlisting{q1.m}
\end{minipage}

\end{document}